\documentclass{beamer}

\usepackage{igo}

\usetheme{Copenhagen}
\usecolortheme{default}

\title{Exploring Positional Linear Go}
\author[Weninger, Hayward]
{Noah Weninger \and Ryan Hayward}
\institute[University of Alberta]
{
  Department of Computing Science\\
  University of Alberta\\
  Canada
}
\date[ACG 2017]
{Advances in Computer Games, 2017}
\subject{Computing Science}

\begin{document}

    \frame{\titlepage}

    \begin{frame}
        \frametitle{Linear Go}
        Go on an Nx1 board.
        \bigskip
        \white{e2,h2}
        \black{g2,c2}
        \begin{center}
            \showgoban[b2,k2]
        \end{center}
    \end{frame}

    \begin{frame}
        \frametitle{Motivation}
        Cycles appear much more often on a linear board. So the game takes on a very different form.
        No safe territory.
        Predicting the outcome of long and complex cycles is essential to playing well.
    \end{frame}

    \begin{frame}
        \frametitle{On rules}
        We consider Tromp-Taylor with positional superko.
    \end{frame}

    \begin{frame}
        \frametitle{On rules}
        Previous work used ...
    \end{frame}

    \begin{frame}
        \frametitle{On rules}
        Difficulties in solving with positional superko
    \end{frame}

    \begin{frame}
        \frametitle{Definitions}
        \begin{itemize}
            \item \textbf{Position}
            \item \textbf{State}
            \item \textbf{...}
        \end{itemize}
    \end{frame}

    \begin{frame}
        \frametitle{Our Goal}
        Solve the game.

        Determine minimax value for many board sizes and opening moves.
    \end{frame}


    \begin{frame}
        \frametitle{Theorems \& other interesting results}
        Cell 2, telomere, full board, stable board, etc. go in this section
    \end{frame}

    \begin{frame}
        \frametitle{Solver implementation}
    \end{frame}

    \begin{frame}
        \frametitle{Results}
        Empty board scores line up with Erik's work.

        We found scores for all opening moves on sizes up to 1x9.
    \end{frame}

\end{document}
