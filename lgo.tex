\documentclass{beamer}

\usepackage{igo}

\usetheme{Copenhagen}
\usecolortheme{default}

\title{Exploring Positional Linear Go}
\author[Weninger, Hayward]
{Noah Weninger \and Ryan Hayward}
\institute[University of Alberta]
{
  Department of Computing Science\\
  University of Alberta\\
  Canada
}
\date[ACG 2017]
{Advances in Computer Games, 2017}
\subject{Computing Science}

\begin{document}

    \frame{\titlepage}

    \begin{frame}
        \frametitle{Linear Go}
        Go on an $Nx1$ board.
        \bigskip
        % the monster
        \begin{center}
            \cleargoban
            \black{c2,d2}
            \white{f2,h2}
            \showgoban[b2,j2]
        \end{center}
    \end{frame}

    \begin{frame}
        \frametitle{Motivation}
        Solving Go is hard. Maybe $Nx1$ Go is easier and can lead to intuition about NxM Go?
        \begin{center}
            \textbf{Not necessarily. Some unique challenges arise.}
        \end{center}
    \end{frame}
    \begin{frame}
        \frametitle{Motivation}
        \begin{itemize}[<+->]
            \item Determine size of game graph vs number of cells on board.
            \item Something about safety... be careful to word this accurately.
            \item Cycles appear much more often on a linear board: ko rules play a more prominent role.
            \item Predicting the outcome of long and complex cycles is essential to playing well.
        \end{itemize}
    \end{frame}

    \begin{frame}
        \frametitle{Definitions}
        \begin{itemize}[<+->]
            \item \textbf{Position} A board configuration.
            \item \textbf{State} A position, along with a set of positions that have been previously visited, and a flag to indicate whether the last move was a pass.
            \item \textbf{...}
        \end{itemize}
    \end{frame}

    \begin{frame}
        \frametitle{On rules}
        We consider Tromp-Taylor rules with positional superko, no suicide and no komi.
        \bigskip
        \begin{center}
            \cleargoban
            \showgoban[b2,h2]\\\medskip
            \pause
            \black{c2}
            \showgoban[b2,h2]\\\medskip
            \pause
            \white{d2}
            \showgoban[b2,h2]\\\medskip
            \pause
            \black{f2}
            \showgoban[b2,h2]\\\medskip
            \pause
            \white{g2}
            \showgoban[b2,h2]\\\medskip
            \pause
            \black{e2}
            \clear{d2}
            \showgoban[b2,h2]\\\medskip
            \pause
            \white{d2}
            \clear{e2,f2}
            \showgoban[b2,h2]\\\medskip
            \pause
            \black[\igocross]{f2}
            \black[\igotriangle]{b2}
            \showgoban[b2,h2]\\\medskip
        \end{center}
    \end{frame}

    \begin{frame}
        \frametitle{On rules}
        Previous work by Erik van der Werf, \textit{Solving Go for Rectangular Boards},
        describes a solver \textit{Migos} which uses different rules
        and solves $Nx1$ Go for $N$ up to 12.\\\medskip
        The rule set used by \textit{Migos} uses situational superko, and ... TODO
    \end{frame}

    \begin{frame}
        \frametitle{On rules}
        Difficulties in solving with positional superko
    \end{frame}

    \begin{frame}
        \frametitle{Our Goal}
        Solve the game.

        Determine minimax value for many board sizes and opening moves.
    \end{frame}


    \begin{frame}
        \frametitle{Theorems \& other interesting results}
        Cell 2, telomere, full board, stable board, etc. go in this section
    \end{frame}

    \begin{frame}
        \frametitle{Solver implementation}
    \end{frame}

    \begin{frame}
        \frametitle{Results}
        Empty board scores line up with Erik's work.

        We found scores for all opening moves on sizes up to 1x9.

        \begin{figure}[th]
        \begin{tabular}{|c|ccccccccc|} \hline
        n & \multicolumn{9}{c|}{minimax value by 1st-move location} \\ \hline
        2 & -2 & - &&&&&&& \\
        3 & -3 & 3 & - &&&&&& \\
        4 & -4 & 4 & - & - &&&&& \\
        5 & -5 & 0 & 0 & - & - &&&& \\
        6 & -6 & 1 & -1 & - & - & - &&&\\
        7 & -7 &  2 & -2 & 2 & - & - & - &&\\
        8 & -3 &  3 & -1 & 1 & - & - & - & - &\\
        9 & -4 & 0  & -1  & 0 & 0 & - & - & - & -\\ \hline
        \end{tabular}
        \caption{Missing entries follow by left-right symmetry.}\label{fig:1st_move_value}
        \label{fig:mmx}
        \end{figure}
    \end{frame}

\end{document}
