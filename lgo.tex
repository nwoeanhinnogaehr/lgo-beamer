\documentclass{beamer}

\usepackage{igo}

\usetheme{Copenhagen}
\usecolortheme{default}

\title{Exploring Positional Linear Go}
\author[Weninger, Hayward]
{Noah Weninger \and Ryan Hayward}
\institute[University of Alberta]
{
  Department of Computing Science\\
  University of Alberta\\
  Canada
}
\date[ACG 2017]
{Advances in Computer Games, 2017}
\subject{Computing Science}

\begin{document}

    \frame{\titlepage}

    \begin{frame}
        \frametitle{Linear Go}
        Go on an Nx1 board.
        \bigskip
        % the monster
        \begin{center}
            \cleargoban
            \black{c2,d2}
            \white{f2,h2}
            \showgoban[b2,j2]
        \end{center}
    \end{frame}

    \begin{frame}
        \frametitle{Motivation}
        \begin{itemize}[<+->]
            \item There is no safe territory.
            \item Cycles appear much more often on a linear board.
            \item Predicting the outcome of long and complex cycles is essential to playing well.
            \item todo...
        \end{itemize}
    \end{frame}

    \begin{frame}
        \frametitle{On rules}
        We consider no-suicide Tromp-Taylor rules with positional superko.
        \bigskip
        \begin{center}
            \cleargoban
            \showgoban[b2,h2]\\\medskip
            \pause
            \black{c2}
            \showgoban[b2,h2]\\\medskip
            \pause
            \white{d2}
            \showgoban[b2,h2]\\\medskip
            \pause
            \black{f2}
            \showgoban[b2,h2]\\\medskip
            \pause
            \white{g2}
            \showgoban[b2,h2]\\\medskip
            \pause
            \black{e2}
            \clear{d2}
            \showgoban[b2,h2]\\\medskip
            \pause
            \white{d2}
            \clear{e2,f2}
            \showgoban[b2,h2]\\\medskip
            \pause
            \black[\igocross]{f2}
            \showgoban[b2,h2]\\\medskip
        \end{center}
    \end{frame}

    \begin{frame}
        \frametitle{On rules}
        Previous work used ...
    \end{frame}

    \begin{frame}
        \frametitle{On rules}
        Difficulties in solving with positional superko
    \end{frame}

    \begin{frame}
        \frametitle{Definitions}
        \begin{itemize}[<+->]
            \item \textbf{Position} A board configuration.
            \item \textbf{State} A position, along with a set of positions that have been previously visited, and a flag to indicate whether the last move was a pass.
            \item \textbf{...}
        \end{itemize}
    \end{frame}

    \begin{frame}
        \frametitle{Our Goal}
        Solve the game.

        Determine minimax value for many board sizes and opening moves.
    \end{frame}


    \begin{frame}
        \frametitle{Theorems \& other interesting results}
        Cell 2, telomere, full board, stable board, etc. go in this section
    \end{frame}

    \begin{frame}
        \frametitle{Solver implementation}
    \end{frame}

    \begin{frame}
        \frametitle{Results}
        Empty board scores line up with Erik's work.

        We found scores for all opening moves on sizes up to 1x9.
    \end{frame}

\end{document}
